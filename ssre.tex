% biber
\documentclass[conference]{IEEEtran}

\usepackage[pdftex]{graphicx}
\usepackage{biblatex}
\graphicspath{{../pdf/}{../jpeg/}}

\usepackage{amsmath}
\usepackage{url}
% correct bad hyphenation here
\hyphenation{op-tical net-works semi-conduc-tor}

\begin{document}
\title{Using Scapy for \\ Man-in-the-Middle Attacks}

\author{
    \IEEEauthorblockN{Ana Rita Martinho}
    \IEEEauthorblockA{Department of Electrical and Computer Engineering\\
        Faculty of Engineering of University of Porto\\
        up201709727@fe.up.pt}
    \and
    \IEEEauthorblockN{Gonçalo Xavier}
    \IEEEauthorblockA{Department of Electrical and Computer Engineering\\
        Faculty of Engineering of University of Porto\\
        up201604506@fe.up.pt}
}

\maketitle

\begin{abstract}
    A Man-in-the-Middle(MITM) attack is a type of cyberattack where an attacker
    secretly relays and possibility alters the communication between 2 
    unsuspecting parties.
    Most recent cryptographic protocols include some form of endpoint 
    authentication specifically to prevent MITM attacks. 
    
    The goal of this project is to demonstrate, using scapy, why such 
    measurements are needed.
    Scapy is a python module that enables a user to send, sniff and forge 
    network packets, as such it provides easy methods to the types of 
    functionalities needed to run a MITM attack.
    
    4 different protocols were analyzed (FTP, SNMP, Telnet and SMTP).


\end{abstract}

\section{Introduction}

This paper will focus on the work developed for the main project of the 
Security for Systems and Networks course.

\section{Related work}
You should identify and analyze related work here. 


\section{Solution}

Here you should describe the solution.

\section{Evaluation}

The evaluation results go here.

% !TeX root = ssre.tex
\section{Conclusion}
\label{sec:Concl}

Through this project we were able to have a 1st hand experience with the 
procedures involved in a MITM attack, the security flaws present in the types
of services that fall victim to them and their glaring consequences.

In a general fashion, most of the compromised services analysed would have been 
safe if their communications were implemented on top of 
\underline{encrypted channels} - rendering the attackers interpretation of the 
detected information useless.
In fact, encryption also plays an extremely important role against the
\textbf{Evesdropping} step mentioned in section \ref{sec:Intro}, since no 
identifying content can be detected by the attacker if all traffic is encrypted.

Prevention of the \textbf{Positioning} attacker's step can be done by not 
accepting any unsolicited ARP responses or by some sort of cross-checking of 
ARP responses involving the DHCP server such that both dynamic and static IP
addresses are verified.

In summary, this project showed us how important concepts like encryption 
really are in a practical way.
A special high note should be given to Scapy, not only for its ease of features 
(and implementation), but also its documentation and available resources.


\begin{thebibliography}{1}

\bibitem{IEEEhowto:kopka}
H.~Kopka and P.~W. Daly, \emph{A Guide to \LaTeX}, 3rd~ed.\hskip 1em plus
  0.5em minus 0.4em\relax Harlow, England: Addison-Wesley, 1999.

\end{thebibliography}

% that's all folks
\end{document}


