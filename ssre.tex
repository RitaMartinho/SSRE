
\documentclass[conference]{IEEEtran}

\usepackage[pdftex]{graphicx}
\graphicspath{{../pdf/}{../jpeg/}}

\usepackage{amsmath}
\usepackage{url}
% correct bad hyphenation here
\hyphenation{op-tical net-works semi-conduc-tor}

\begin{document}
\title{Using Scapy for \\ Man-in-the-Middle Attacks}

\author{
    \IEEEauthorblockN{Ana Rita Martinho}
    \IEEEauthorblockA{Department of Electrical and Computer Engineering\\
        Faculty of Engineering of University of Porto\\
        up201709727@fe.up.pt}
    \and
    \IEEEauthorblockN{Gonçalo Xavier}
    \IEEEauthorblockA{Department of Electrical and Computer Engineering\\
        Faculty of Engineering of University of Porto\\
        up201604506@fe.up.pt}
}

\maketitle

\begin{abstract}
    A Man-in-the-Middle(MITM) attack is a type of cyberattack where an attacker
    secretly relays and possibility alters the communication between 2 
    unsuspecting parties.
    Most recent cryptographic protocols include some form of endpoint 
    authentication specifically to prevent MITM attacks. 
    
    The goal of this project is to demonstrate, using scapy, why such 
    measurements are needed.
    Scapy is a python module that enables a user to send, sniff and forge 
    network packets, as such it provides easy methods to the types of 
    functionalities needed to run a MITM attack.
    
    4 different protocols were analyzed (FTP, SNMP, Telnet and SMTP).


\end{abstract}


\section{Introduction}

Here you should provide the context of this work and describe the problem you are solving. 

\section{Related work}
You should identify and analyze related work here. 

\section{Solution}

Here you should describe the solution.

\section{Evaluation}

The evaluation results go here.



\section{Conclusion}
The conclusion goes here.




% trigger a \newpage just before the given reference
% number - used to balance the columns on the last page
% adjust value as needed - may need to be readjusted if
% the document is modified later
%\IEEEtriggeratref{8}
% The "triggered" command can be changed if desired:
%\IEEEtriggercmd{\enlargethispage{-5in}}

% references section

% can use a bibliography generated by BibTeX as a .bbl file
% BibTeX documentation can be easily obtained at:
% http://mirror.ctan.org/biblio/bibtex/contrib/doc/
% The IEEEtran BibTeX style support page is at:
% http://www.michaelshell.org/tex/ieeetran/bibtex/
%\bibliographystyle{IEEEtran}
% argument is your BibTeX string definitions and bibliography database(s)
%\bibliography{IEEEabrv,../bib/paper}
%
% <OR> manually copy in the resultant .bbl file
% set second argument of \begin to the number of references
% (used to reserve space for the reference number labels box)
\begin{thebibliography}{1}

\bibitem{IEEEhowto:kopka}
H.~Kopka and P.~W. Daly, \emph{A Guide to \LaTeX}, 3rd~ed.\hskip 1em plus
  0.5em minus 0.4em\relax Harlow, England: Addison-Wesley, 1999.

\end{thebibliography}




% that's all folks
\end{document}


