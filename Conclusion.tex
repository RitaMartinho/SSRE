% !TeX root = ssre.tex
\section{Conclusion}
\label{sec:Concl}

Through this project we were able to have a 1st hand experience with the 
procedures involved in a MITM attack, the security flaws present in the types
of services that fall victim to them and their glaring consequences.

In a general fashion, most of the compromised services analysed would have been 
safe if their communications were implemented on top of 
\underline{encrypted channels} - rendering the attackers interpretation of the 
detected information useless.
In fact, encryption also plays an extremely important role against the
\textbf{Evesdropping} step mentioned in section \ref{sec:Intro}, since no 
identifying content can be detected by the attacker if all traffic is encrypted.

Prevention of the \textbf{Positioning} attacker's step can be done by not 
accepting any unsolicited ARP responses or by some sort of cross-checking of 
ARP responses involving the DHCP server such that both dynamic and static IP
addresses are verified.

In summary, this project showed us how important concepts like encryption 
really are in a practical way.
A special high note should be given to Scapy, not only for its ease of features 
(and implementation), but also its documentation and available resources.
