\section{Introduction}
\label{sec:Intro}
This paper will focus on the work developed for the main project of the 
Security for Systems and Networks course.

This projects aims to demonstrate the different procedures used to carry out a 
MITM attack and their potential outcomes.

3 Linux Virtual Machines (VM) were used, 2 simulating normal users of the 
analyzed protocols - victims - and 1 running scapy, simulating the potential 
attacker.

The general flow of a MITM attack is as follows: \textbf{Sniffing/Evesdropping}
- the attacker detects some kind of communication or vulnerable victims - 
\textbf{Positioning} - the attacker position itself between both victims -
\textbf{Exploitation} - the attacker stores or even alters the collected 
information for nefarious actions.

\subsection{Analyzed Protocols}

\subsubsection{FTP}

FTP stands for File Tranfer Protocol and it is a standard network protocol used in computer file transfer between 2 nodes in a network, using a hierarchical architecture in a client-server model fashion (the server being the node storing the files and the client the node requesting it). 
Clients can authenticate themselves in an \textit{anonymous} way - if the server is prepared for that - or using a username and password that is sent in \textbf{clear text}. 
With that in mind and knowing that FTP server uses well-known reply codes, (such as \textit{ 230 - User logged in, proceed.} ) an attacker can easily store and use valid credentials to access the server, impersonating a valid client.  

\subsubsection{SNMP}

SNMP stands for Simple Network Management Protocol and it is used for collecting, organizing and changing information about managed devices on Layer 3 networks
. SNMP defines structured management data using a Management Information Base (MIB). 
This database is hierarchical (tree-structured) and each entry is addressed through an object identifier (OID). 
SNMP follows a client-server model in which the servers (\textit{managers}) collect and process information about devices on a network. 
The clients (\textit{agents}) are any type of device in a network that can be monitored, such as computers, switches, routers, printers, etc. 
There are 3 main SNMP commands: - \textit{set}: the manager sets some value to an OID value available on the agent; \textit{get}: the manager queries the agent about some OID value and \textit{trap}: the agent sends unsolicited information about some OID value to the manager. 
These commands require the use of a \textit{community string}, representing a somewhat password. 
In version 1 and 2 of the protocol this community string is sent in \textbf{clear text} and so, any attacker, sniffing a SNMP communication can \textit{set} or \textit{get} any OID value using the retrieved community string. 
SNMPv3 corrects this lack of security. 

\subsubsection{Telnet}

Telnet is a server-client application protocol that is typically used to open a command line on a remote computer. 
It uses a bidirectional interactive text-oriented communication facility. 
When a user tries to access the remote computer, they are prompted to enter their username and password combination. 
Telnet, by default, \textbf{doesn't encrypt} any type of data, neither the pass and user combination nor the data sent over the connection, presenting a huge lack of security, especially when this protocol is deployed over the Internet - an attacker can  eavesdrop on the communications and use the gathered information for malicious purposes. 
In fact, Telnet is becoming deprecated in favor of the SSH protocol.


\subsubsection{SMTP}

SMTP stands for Simple Mail Transfer Protocol and it is used for sending and receiving e-mail. 
SMTP also follows a client-server model, with the server being the application that its primary purpose is to send, receive, and/or relay outgoing mail between email senders and receivers. 
In other works, a client email sender, writes the email body and subject and specifies the receiver. 
This information is sent to the SMTP server which is responsible to relay this email to the appropriate receiver, if possible. 
This information is sent using \textbf{clear text} and any attacker in the middle of the client-server communication can access the contents of the email, compromising a mostly desired privacy. 
Similarly to FTP, SMTP server also uses reply codes, which increases the facility of the attacker to know which type of data is being sent.

